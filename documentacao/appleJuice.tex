\documentclass[a4paper,12pt]{article}
\usepackage[utf8]{inputenc}
\usepackage[brazil]{babel}
\usepackage{amsmath}
\usepackage{setspace}
\usepackage{indentfirst}

\setlength{\parindent}{0.62cm}

\title{Principais componentes da placa Apple Juice e seu respectivo funcionamento}
\author{}
\date{}

\begin{document}

\maketitle

\section{Temporizador/gerador de clock NE555 em modo astável}

O circuito integrado 555, quando configurado em modo astável, atua como um gerador de onda quadrada contínua, alternando automaticamente entre níveis lógicos alto (1) e baixo (0), sem a necessidade de um sinal de entrada externo.

Esse comportamento pode ser interpretado como um processo periódico contínuo, no qual há alternância de estados em função do tempo.

O funcionamento baseia-se em três componentes principais: dois resistores e um capacitor. O capacitor realiza ciclos sucessivos de carga e descarga:
\begin{itemize}
    \item Quando a tensão no capacitor atinge $\frac{2}{3}$ da tensão de alimentação, a saída do circuito é comutada para nível lógico baixo.
    \item Quando a tensão decai até $\frac{1}{3}$ da tensão de alimentação, a saída retorna ao nível lógico alto.
\end{itemize}

\vspace{0.5cm}

Os intervalos de tempo associados aos níveis lógico alto e baixo são descritos pelas seguintes equações:

\vspace{0.3cm}

\begin{equation}
t_{high} = \ln(2)\,(R_1 + R_2)\,C
\end{equation}

\vspace{0.4cm}

\begin{equation}
t_{low} = \ln(2)\,R_2\,C
\end{equation}

\vspace{0.5cm}

A frequência do sinal gerado é dada por:

\vspace{0.3cm}

\begin{equation}
f = \frac{1}{t_{high} + t_{low}}
\end{equation}

\vspace{0.5cm}

\section{Contador Johnson 4017}

O circuito integrado 4017 é um contador Johnson de década com dez saídas distintas. A cada pulso de clock recebido, apenas uma saída é ativada em nível lógico alto, enquanto as demais permanecem em nível baixo.

A sequência de ativação das saídas ocorre de forma cíclica:
\begin{center}
Q0 $\rightarrow$ Q1 $\rightarrow$ Q2 $\rightarrow$ ... $\rightarrow$ Q9 $\rightarrow$ reinício
\end{center}

Esse comportamento pode ser interpretado como uma máquina de estados finitos, na qual cada estado corresponde a uma saída ativa. A cada pulso de clock, o sistema transita para o próximo estado da sequência.

O reinício da contagem pode ocorrer automaticamente após o último estado ou por meio de um sinal externo aplicado ao pino de reset. Na placa Apple Juice, o pino Q4 foi conectado direto ao reset para que quando ele ficasse ativo, o sistema reiniciasse.

\section{Contador com display 4026}

O circuito integrado 4026 consiste em um contador decimal acoplado a um decodificador para display de sete segmentos.

A cada pulso de clock recebido, o circuito incrementa seu valor interno, avançando sequencialmente de 0 até 9.

Cada valor é automaticamente convertido em sinais elétricos que controlam diretamente um display de sete segmentos, permitindo a exibição dos números sem a necessidade de circuitos adicionais.

Após atingir o valor máximo (9), o contador retorna ao estado inicial (0), reiniciando o ciclo.

\section{Resumo}

\begin{itemize}
    \item O 555 é responsável pela geração de um sinal periódico de clock.
    \item O 4017 realiza a distribuição sequencial desse sinal entre múltiplas saídas.
    \item O 4026 efetua a contagem dos pulsos e a exibição direta em um display.
\end{itemize}

\section{Referências}

\begin{itemize}
    \item Texas Instruments. \textit{NE555 Precision Timer Datasheet}.
    \item Texas Instruments. \textit{CD4017B Decade Counter/Divider Datasheet}.
    \item Texas Instruments. \textit{CD4026B Decade Counter/7-Segment Display Driver Datasheet}.
\end{itemize}

\end{document}
